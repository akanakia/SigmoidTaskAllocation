\documentclass[12pt]{article}

\usepackage{graphicx}
\usepackage{amsmath}
\usepackage{amsfonts}
\usepackage{amssymb}
\usepackage{subcaption}
\usepackage{pdfpages}

\def\td{\mathbf{t}}   % response-threshold value

\begin{document}
\includepdf[pages={-}]{nature-text.pdf}

\newpage
\subsection*{Figures}
\begin{figure*}[ht!]
    \centering\begin{subfigure}[t]{0.5\textwidth}
        \centering
        \includegraphics[width=1\textwidth]{figures/firefighting.png}
        \caption{Robot Firefighting}
    \end{subfigure}%
    ~ 
    \begin{subfigure}[t]{0.5\textwidth}
        \centering
        \includegraphics[width=1\textwidth]{figures/bankrun.png}
        \caption{Bank Run}
    \end{subfigure}
    \begin{subfigure}[t]{1\textwidth}
        \centering
        \includegraphics[width=1\textwidth]{figures/foraging.png}
        \caption{Ant Foraging}
    \end{subfigure}    
    \caption{Robotic fire fighting, ant foraging, and bank run scenarios presented as global games. Each player's imperfect estimate of the task is represented by $x_i$, comprising of the global stimulus parameter $\tau$ and noisy sensor measurements $\eta_i$. In the robot firefighting scenario $\tau$ is representative of the magnitude of the fire, while in the case of a bank run $\tau$ is indicative of an agent's current level of trust in the nation's economy. For the ant foraging scenario $\tau$ represents an ant's willingness to take part in the foraging task based on a number of internally measured parameters such as the distance to the food source ($t_t$), the wait time to deliver food ($t_w$), and the food stores currently at the nest ($s$), among others.
%Whereas $\tau$ is representative of the minimum of number of agents needed to extinguish a fire when taking simultaneous action, it can also be representative for a cumulation of different observations such as during ant foraging, which depends on the food available in the nest, the distance to the source, and the size of the source, or the global economy such as prior to a bank run.
}    
\end{figure*}

\newpage
\begin{figure}[!ht]
	\centering\includegraphics[width=0.7\columnwidth]{figures/thm2fig.png}
	\centering\caption{Visualization of Theorem~2 as $N_{rel}$ estimates $\Phi(\cdot)$. The plot was generated by running Eqn.~1 10,000 times for each point in $\hat{\tau} = 1$ to $10$ in increments of $0.1$. $n = 10$, $\td = 5$ and $x_i = \hat{\tau} + \eta_i$ ($\eta_i \sim\mathcal{N}(0, \sigma^2)$). Each curve in the plot is generated by sweeping $\sigma^2 = \{0, 0.1, 1, 2, 10\}$, with $\sigma^2 = 0$ being a step-function and $\sigma^2 = 10$ having the \emph{flattest} slope.}\label{fig:thm2fig}
\end{figure}
\end{document}