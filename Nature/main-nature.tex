%% Template for a preprint Letter or Article for submission
%% to the journal Nature.
%% Written by Peter Czoschke, 26 February 2004
%%
\documentclass{nature}

%% make sure you have the nature.cls and naturemag.bst files where
%% LaTeX can find them
\bibliographystyle{naturemag}

\title{Mathematical Origins of the Sigmoid Response-Threshold Function}

%% Notice placement of commas and superscripts and use of &
%% in the author list

\author{Anshul Kanakia$^{1,2}$, Nikolaus Correll$^{1,2}$ \& Behrouz Touri$^{1,3}$}


\begin{document}

\maketitle

\begin{affiliations}
 \item College of Engineering and Applied Sciences, University of Colorado, Boulder, USA. 
 \item Dept. of Computer Science.
 \item Dept. of Electrical, Computer \& Energy Engineering.
\end{affiliations}

\begin{abstract}
Abstract goes here\ldots
\end{abstract}

%% Paper content goes here. Sections are not allowed in Letters to Nature
Collaboration is a fundamental necessity in any society. Tasks such as foraging, farming, hunting, and construction that were essential building blocks of human civilization greatly benefited from collaboration between multiple individuals. Social insects have evolved to facilitate collaboration in a decentralized manner. The problem of allocating tasks to a group of agents is solved in a completely distributed fashion within insect swarms such as ant colonies and bee hives. While a central queen organism often exists in insect colonies, this organism's duty is purely reproductive in nature, not administrative.

%%% WHAT IS THE KEY PROBLEM WITH COLLABORATION? AGREEING ON WHEN TO DO IT REQUIRES A COMMON STIMULULUS AND COORDINATION
This sheds light on the key complexity for collaboration; a group must perceive a common stimulus and coordinate to complete a complex task without directives from a leader. 

Ethologists have studied the problem of distribution of labor in social insects extensively and have found that a class of models called response-threshold models explain observed behavior very well. Swarm roboticists have adopted this model of distributed task-assignment and adapted it for use with artificially engineered sys tems. While extensive phenomenological evidence exists supporting the agreement between response-threshold models and successful task-assignment, there has so far been no logical argument for their extensive use. 
%%% THE PROBLEM HAS NOT ONLY BEEN STUDIED BY ETHOLOGISTS FOR ANTS BUT ALSO BY NEUROLOGISTS AND ECONOMISTS (GAME THEORY). WHO ELSE STUDIES TASK ALLOCATION IF IT IS REALLY SO IMPORTANT?

%%% DONT FRAME THE PAPER AROUND THE SOLUTION "RESPONSE THRESHOLD", BUT AROUND THE COMMON SETUP: A NUMBER OF AGENTS MUST AGREE UPON SOMETHING BUT CANNOT TALK TO EACH OTHER, JUST OBSERVE A COMMON SIGNAL WITH MAYBE NOISE. 
This is the scenario setup. There are a group of individual agents attempting to perform some collaborative task. Each agent has some internal representation of the magnitude of the task but does not share this information with the other agents. All agents decide, for themselves, whether or not to take part in the collaborative task. When a critical mass of agents willing to take part in the task is achieved, the task is successfully attempted, otherwise the attempt automatically fails.

We have seen situations like this in a number of different fields including ethology, neurology, sociology, economics, and computer science. While collaboration in social organisms has been a topic of study for centuries, the underlying mechanisms of this process still elude researchers. A number of models have been proposed including the response threshold model (find others). Of these models, response threshold models have become popular due to their simplicity and accuracy with which they successfully describe collaborative task behavior in multi-agent systems. Roboticists have started using these models to engineer swarm systems.

%%% WHAT ARE THE COMMONALITIES AMONG ALL THE CONSENSUS PROBLEMS / WHAT CLASS OF PROBLEMS ARE WE INTERESTED IN. THIS SHOULD MATCH THE GAME THEORY DEFINITION. THERE ARE MANY. WHERE HAS THE RESPONSE THRESHOLD ALREADY BEEN OBSERVED AND WHERE IS IT USED? ETHOLOGY AND ROBOTICS. THEY ARE DIFFERENT THAN WHAT GAME THEORY ASSUMES, HERE IS WHY: NOISE. 
Raafat et al. (2009) discuss the underlying control mechanism for distributed coordination in a multi-agent system. They split the study of this mechanism into two levels; transmission-based and pattern-based control. 

% MAIN CLAIM: THERE ARE COUNTLESS SYSTMES THAT MUST RESPONSE THRESHOLDS, BECAUSE THEY ARE THE BEST WAY TO DO IT IN THIS CASE.

% Notes on cited papers:
The theory of global games was first introduced by Carlsson and Van Damme in \cite{Carlsson1993}.

\cite{Gordon1996}

\cite{Theraulaz1998}

\cite{Martinoli1999}

\cite{Bonabeau2000}

\cite{Krieger2000}

\cite{Kube2000}

A survey of the application of global games to economic settings is available in \cite{Morris2000}.

\cite{Pynadath2002}

\cite{Conradt2003}

\cite{Mataric2003}

\cite{Gerkey2003}

\cite{Gerkey2004}

\cite{Conradt2005}

\cite{Raafat2009}

\cite{Yoshida2010}

\cite{Suzuki2015}




%% Put the bibliography here, most people will use BiBTeX in
%% which case the environment below should be replaced with
%% the \bibliography{} command.
\bibliography{references}

\begin{addendum}
 \item Put acknowledgements here.
 \item[Competing Interests] The authors declare that they have no
competing financial interests.
 \item[Correspondence] Correspondence and requests for materials
should be addressed to Dr. Nikolaus Correll.~(email: nikolaus.correll@colorado.edu).
\end{addendum}

%%
%% TABLES
%%
%% If there are any tables, put them here.
%%

\end{document}
