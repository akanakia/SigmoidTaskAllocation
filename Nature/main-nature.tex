%% Template for a preprint Letter or Article for submission
%% to the journal Nature.
%% Written by Peter Czoschke, 26 February 2004
%%
\documentclass{nature}
%% make sure you have the nature.cls and naturemag.bst files where
%% LaTeX can find them
\bibliographystyle{naturemag}

%% Notice placement of commas and superscripts and use of &
%% in the author list
\usepackage{amsthm}
\newtheorem{theorem}{Theorem}
\newtheorem{lemma}{Lemma}
\newtheorem{claim}{Claim}
\newtheorem{corollary}{Corollary}
\newtheorem{proposition}{Proposition}
\usepackage{amsmath}
\usepackage{amsfonts}
\usepackage{amssymb}
\usepackage{enumerate}
\usepackage{mathtools}
\usepackage{framed}

\usepackage{lineno}

% ----------- CUSTOM MACROS ----------------
\DeclareMathOperator*{\argmin}{\arg\!\min\>}
\newcommand{\amin}[1]{\underset{#1}\argmin}
\DeclareMathOperator*{\argmax}{\arg\!\max\>}
\newcommand{\amax}[1]{\underset{#1}\argmax}
\DeclarePairedDelimiter\ceil{\lceil}{\rceil}
\DeclarePairedDelimiter\floor{\lfloor}{\rfloor}

\newcommand{\vnorm}[1]{\left|\left|#1\right|\right|}
\newcommand{\D}[2]{\frac{d#1}{d#2}}
\newcommand{\PD}[2]{\frac{\partial #1}{\partial #2}}
\newcommand{\V}[1]{\mathbf{#1}}
\newcommand{\ubar}[1]{\underline{#1}}

\def\a{\mathbf{a}}    % Action profile
\def\Z{\mathbb{Z}}    % Integers
\def\R{\mathbb{R}}    % Reals
\def\N{\mathcal{N}}   % Naturals
\def\td{\mathbf{t}}   % response-threshold value
\def\sig{\mathcal{S}} % Sigmoid function

\def\citerq{$^\text{citation reqd.}$}

% ----------------- TITLE AND AUTHORS --------------------
\title{An Equilibrium strategy for task allocation in systems with limited communication}
\author{Anshul Kanakia$^{1,2}$, Behrouz Touri$^{1,3}$ \& Nikolaus Correll$^{1,2}$}

\begin{document}
\maketitle
\begin{affiliations}
 \item College of Engineering and Applied Sciences, University of Colorado, Boulder, USA. 
 \item Dept. of Computer Science.
 \item Dept. of Electrical, Computer \& Energy Engineering.
\end{affiliations}

% --------------- PAPER BODY --------------------
\linenumbers
\begin{abstract}
Task allocation in systems subject to communication constraints is ubiquitous in nature at all scales, ranging from cellular systems\cite{Yoshida2010, Suzuki2015} and social insects\cite{Robinson1987, Gordon1996, Bonabeau1998, Theraulaz1998} to large animal herds\cite{Conradt2003, Conradt2005} and human society\cite{Raafat2009}. Since the ability to communicate decisions such as during voting\cite{Conradt2003} increases with agent complexity, inter-agent communication might not always be possible. This may be either due to physical limitations at the agent level (cellular/insect systems) or properties of the task itself (adversary behaviour in humans). Here we show that formalizing this class of problems as a global game, a concept from the field of game theory, reveals that a simple threshold policy leads to a Bayesian Nash Equilibrium despite the absence of communication between agents. While this result provides a hypothesis about the inner workings of a wide range of systems in which limited communication between agents is likely, it provides a formal explanation for threshold-based task allocation in social insects. In particular, we show how noise in the perception apparatus of such systems leads to commonly observed sigmoid response threshold functions that control the trade-off between exploration and exploitation\cite{Bonabeau1997} in natural systems and can be used to design engineered systems ranging from swarm robotics\cite{Martinoli1999, Krieger2000, Kube2000, Mataric2003, Gerkey2004} to smart composites\cite{McEvoy2015}, made of computational elements with very low complexity. 
\end{abstract}

Consider a group of agents performing a task contributing to a common metric, which we refer to as a concurrent benefit. This benefit is related to a stimulus $\tau$ that can be observed by all agents, albeit subject to sensing noise. Agents do not share any information. All agents decide, for themselves, whether or not to engage in the task. When a critical mass of agents willing to take part in the task is achieved, the task is successfully attempted, otherwise the attempt automatically fails. 
%When a critical mass of agents taking part in the task is achieved, a strong non-linear (to the number of agents) response to the original stimulus is observed within the system.

Situations like this arise in a number of different fields including neurology\cite{Yoshida2010, Suzuki2015}, ethology\cite{Robinson1987, Gordon1996, Bonabeau1998, Theraulaz1998}, sociology\cite{Raafat2009}, economics\cite{Morris2000}, and robotics\cite{Martinoli1999, Krieger2000, Kube2000, Pynadath2002, Gerkey2003, Mataric2003, Gerkey2004, Kanakia2014}. All of these multi-agent scenarios share the common notion of a joint action or response to a commonly observed stimulus. The task can take on many forms ranging from neurons simply firing in concert, collective decision problems like flocking, herd grazing and colony defense to individual actions based on the environment and other agents' beliefs like foraging, bank runs, and political revolutions. 

%In all these cases, there exists a task or action that demands completion which in turn requires resources. We introduce parameter $\tau$ that empirically describes this property.
In the case of a bank run\cite{Morris2000}, $\tau$ is an aggregate parameter that represents the strength of the economy of a nation. In the case of social insects foraging for food\cite{Bonabeau1996, Theraulaz1998, Krieger2000}, $\tau$ represents a number of environmental cues such as the (imperfect) measurement of food stores in a colony, pheromone levels\cite{Robinson1987} or the waiting time for food transfer from one agent to another\cite{Seeley1989}. A complex combination of these internal and external cues\cite{Gordon1996} temper an agent's perception of the magnitude of a task. In an engineering context, $\tau$ can be seen as the magnitude of a fire (heat intensity and area covered) as sensed by a robot using on-board instruments in an automated firefighting scenario\cite{Kanakia2014}. 

The above examples, sometimes driven by adversial or collaborative behavior, for example during a bank run or firefighting, respectively, seem to fall into orthogonal classes at first. 
%The firefighting and bank run examples may at first seem orthogonal. During the firefighting scenario agents act collaboratively to achieve a common goal while adversarial behaviour is observed in humans during an economic crisis. 
Both scenarios, however, share the notion that to be successful an agent not only needs to assess the magnitude of the task itself but also the likelihood of the other agents to act. We build up on results from global games\cite{Carlsson1993} to show that the observed behaviour in both cases can be effectively emulated by assuming that each agent makes their individual decision on whether or not to perform a task based on some internal threshold value which is compared to their noisy estimates of the collective task's magnitude $\tau$. This property was shown for the canonical bank run example\cite{Morris2000}. Here, agents decide when to withdraw their assets from banks based on their own noisy estimate of the economy together with a simple threshold.  While the classical global game assumes each agent must predict the other agents' behaviour, it turns out that agents can reach an equilibrium without this capacity. This, and the fact that agents do not need to communicate, makes this approach widely applicable to a wide range of multi-agent systems.

We formalize the task allocation problem as a global game to mathematically prove the existence of Bayesian Nash Equilibria (BNE) when a threshold based strategy is employed to solve the problem. Consider a set of $n$ agents and suppose that each agent has an action set $A_i=\{0,1\}$ where $0$ represents not participating in the task and $1$ represents participating in the task. Every agent is also aware of the total number of other agents, $n$ in the system. For the purpose of analysis we assume the decision to act or not to act is made by all agents at the same time, i.e.\ this is a one-shot game with no notion of time. In swarms of minimalist agents with limited capabilities, the resource required to collaboratively complete a task is invariably quantized into the number of agents attempting to complete that task. We therefore let the magnitude parameter $\tau$ be an (often random) real number which represents how many agents are required to complete a task. We further assume that it belongs to an interval $E=[c,d]$ in $\R$. Finally, we let $u_i:A_i\times\Z^+\times \R\to \R$ be the utility of the $i^{\text{th}}$ agent, where $u_i(a_i,g,\tau)$ is the utility of the $i^{\text{th}}$ agent when $g$ other agents have decided to participate in the task of magnitude $\tau$. In general, the utility of each agent depends on the joint actions of the rest of the agents. However, in our formulation the aggregation of joint actions is simplified down to only the number of agents participating in the activity. The following discussion can be substantially generalized but this form of utility serves the purpose of this study. 

Without loss of generality, the class of tasks discussed in this paper have the following properties:
\begin{enumerate}[a.]
	\item $u_i(1,g,\tau)-u_i(0,g,\tau)$ is an increasing and continuous function of $\tau$ for any $a_i\in A_i$ and $g$. We further assume that $|u_i(1,g,\tau)-u_i(0,g,\tau)|\leq \tau^p$ for some $p\geq 1$.
	\item For extreme magnitude ranges, taking part in the activity is either appealing or repelling, i.e.\ there exists $\underline{\tau},\bar{\tau}\in (c,d)$ with $\underline{\tau}\leq \bar{\tau}$ such that for any $\tau\geq \bar{\tau}$, we have $u_i(1,g,\tau)>u_i(0,g,\tau)$ and for $\tau\leq \underline{\tau}$, the only equilibrium of the game is for all agents to not participate as the attempt would be fruitless.
\end{enumerate}
%
Note that in order to have such a task, we need the above conditions to hold for all the agents, i.e.\ for all $i\in\{1,\ldots,n\}$. An example of a utility function that would satisfy such conditions is a function $u_i(a_i,g,\tau)=a_i(1-e^{-(g+1)}+\tau)$. 

The main challenge in devising task allocation strategies is that the true value of $\tau$ is not easily accessible to the agents. For example, during firefighting\cite{Kanakia2014}, agents have only noisy imperfect information through on-board sensors and their measurements to estimate the magnitude of the task. We model this imperfect knowledge by assuming that agent $i$ observes $x_i=\tau+\eta_i$ where $\eta_i$ is a Gaussian $\mathcal{N}(0,\sigma_i^2)$ random variable, as seen in Fig.~1. Throughout our discussion, we assume that the task magnitude $\tau$ is a Gaussian random variable and is independent of $\eta_1,\ldots,\eta_n$. This analysis is extendable to a larger class of random variables but for the simplicity of the discussion, we consider Gaussian random variables here. Now, the main question is, given an agent's private measurement $x_i$, what is a \emph{sensible strategy} to follow?

First, a \emph{strategy} $s_i$ for the $i^{\text{th}}$ agent is a measurable function $s_i:\R\to A_i$, mapping measurements (observations) to actions. Strategy $s_i$ prescribes what action the $i^{\text{th}}$ agent should take given its own measurement $x_i$. Given this, consider a set of agents with strategies $s_1,\ldots,s_n$. Let us denote the strategies of the $n-1$ agents other than the $i^{\text{th}}$ agent by the vector $s_{-i}=(s_1,\ldots,s_{i-1},s_{i+1},\ldots,s_n)$.  We say that a strategy $s_i$ is a \emph{threshold strategy} if $s_i(x)=\text{step}(x, t_i)$, i.e.\ the step function with a jump from $0$ to $1$ at $t_i$. For the $i^{\text{th}}$ agent, we define the best-response $BR(s_{-i})$ (to the strategies of the other agents) to be a strategy $\tilde{s}$ that for any $x\in \R$:
\begin{align*}\label{eqn:BR}
BR(s_{-i})(x)&=\tilde{s}(x)=\amax{a_i\in A_i} E(u_i(a_i,g,\tau)\mid x_i=x)\\
&=\amax{a_i\in A_i} E(u_i(a_i,\sum_{j\not=i}s_j(x_j),\tau)\mid x_i=x),
\end{align*}
where $E(\cdot \mid x_i)$ is the conditional probability of $u_i$ given the $i^{\text{th}}$ agent's observation. Note that given $x_i$ and the strategies of the other agents $s_{-i}=(s_1,s_2,\ldots,s_{i-1},s_{i+1},\ldots,s_n)$, $\tau$, and $s_j(x_j)$ will be a random variable. In other words, given the $i^{\text{th}}$ agent's observation $x_i$, the observation of the other agents and hence, their actions would be random from the $i^{\text{th}}$ agent perspective.

A strategy profile $s=(s_1,\ldots,s_n)$ is a \emph{sensible strategy}, if it leads to a \emph{Bayesian Nash Equilibrium} (BNE) \cite{Fudenberg1998}, given $s_i=BR(s_{-i})$ for all $i\in \{1,\ldots,n\}$. In other words, with the current strategies of the $n$ agents, no agent has the incentive to deviate from its current strategy.

We now show that any task with  concurrent benefit admits a threshold strategy BNE. Tha is, it is sufficient for the agents to follow a simple algorithm: 
\begin{enumerate}[(i)]
\item Compare your noisy measurement $x_i$ to a threshold value $\td_i$,
\item If the measurement is above $\td_i$ take part in the collaborative task, otherwise hold off.
\end{enumerate}


\nolinenumbers
\begin{figure}
\begin{framed}
\textbf{Box 1}
To show that there exists a sensible threshold policy for the class of tasks with concurrent benefit, we will first show that the best response to threshold policies is a threshold policy, and then we show that there exists an equlibrium consists of threshold policies\cite{Carlsson1993, Morris2000}. Proofs for Lemmas 1 and 2 are provided in the supplementary materials.
\begin{lemma}\label{lemma:thresholdBR}
Let $s=(s_1,\ldots,s_n)$ be a strategy profile consisting of threshold strategies for a task with  concurrent benefit. Let $\tilde{s}_i=BR(s_{-i})$. Then $\tilde{s}_i$ is a threshold policy. 
\end{lemma} 

We can view the best-response of threshold strategies as a mapping from $\R^n$ to $\R^n$ that maps $n$ thresholds of the original strategies to $n$ thresholds of the best-response strategies. Denote this mapping by $L:\R^n\to\R^n$. We now show that this mapping is a continuous mapping.

\begin{lemma}\label{lemma:continuous}
The mapping $L$ that maps the threshold values of threshold strategies to the threshold values of the best-response strategies is a continuous mapping. 
\end{lemma}

Using the previous Lemmas, we can show the existence of a threshold policy for these games. 
\begin{theorem}\label{thrm:mainthrm}
For a concurrent task $T$, suppose that the magnitude parameter $\tau$ is a Gaussian random variable. Also, suppose that $x_i=\tau+\eta_i$ where $\eta_1,\ldots,\eta_n$ are independent Gaussian random variables. Then, there exists a strategy profile $s=(s_1,\ldots,s_n)$ of threshold policies that is a Bayesian Nash Equilibrium.
\end{theorem}
\begin{proof}
By Lemma~\ref{lemma:thresholdBR}, the best response of a threshold policy is a threshold policy and hence, it induces the mapping $L$ from the space of thresholds $\R^n$ to itself. Also, by Lemma~\ref{lemma:continuous}, this mapping is a continuous mapping. Now, if $t_i$ is a sufficiently large threshold, then the second property of the  concurrent benefit tasks implies that the $\tilde{t}_i\leq t_i$ because a large enough measurement $x_i$ implies that agent $i$ itself should take part in the task. Similarly, for sufficiently low threshold $t_i$, we will have $\tilde{t}_i\geq t_i$. Therefore, the mapping $L$ maps a box $[a,b]^n$ to itself, where $a$ is a sufficiently small scalar and $b>a$ is a sufficiently large scalar. Since the box $[a,b]^n$ is a convex closed set, by the Brouwer's fix point theorem \cite{Border1990}, we have that there exists a vector of threshold values $\alpha^*$ such that $\alpha^*=L(\alpha^*)$ and hence, there exists a Bayesian Nash Equilibrium for a concurrent task $T$.
\end{proof}
\end{framed}
\end{figure}

\linenumbers
We show in Box 1 that a simple response threshold is sufficient to achieve a BNE. This insight is consistent with phenomenological observations of social insects, which are believed to use response thresholds for coordination. For example, Robinson\cite{Robinson1987} provides first empirical evidence for hormone regulated response-threshold labor division in honey bees, and similar observations have been made for termites, ants and other social insects \cite{Bonabeau1999,Camazine2001}. Given their prevalence in social insects and humans, it is likely that similar coordination mechanisms are in place to regulate concurrent activities across length-scales. 
%
%A number of other ethology research groups subsequently show similar hormone regulation based task-allocation processes in ants and termites providing further evidence of the ubiquitous nature of the response-threshold process. 
%
Roboticists have started using these models to engineer swarm systems\cite{Bonabeau1996,Theraulaz1998,Krieger2000}. %Bonabeau \cite{Bonabeau1996} introduces the \emph{fixed} response-threshold model for division of labor in social insects which is later extended to incorporate the notion of \emph{dynamic} response-thresholds for task-allocation\cite{Theraulaz1998}.

Yet, observations in ethology suggest sigmoid threshold function\cite{Bonabeau1996}, rather than fixed thresholds as suggested by the game theoretic framework. We argue that this behaviour can be a direct result of using a simple discrete threshold under the influence of perception noise. Indeed, one can show that a sigmoid threshold function is the outcome of deterministic threshold functions on noise observations. When engineering a system, this noise can be controlled and enhanced using computation, allowing artificial agents to trade between exploitation and exploration. 

After showing that discrete thresholds indeed result in a BNE for systems in which agents do not directly communicate, we will now show how sensor noise at the individual level results in a continuous response-threshold phenomena that is prevalent in MAS and social insects. 

Suppose that all the agents share the same utility function $u(a_i,g,\tau)$ and also, assume that the observation noise of the $n$ agents ($\eta_1,\ldots,\eta_n$) are independent and identically distributed (i.i.d) $\mathcal{N}(0,\sigma^2)$ Gaussian random variables. Then, it is not hard to see that there exists a BNE with threshold strategies that have the same threshold value $\td$\cite{Morris2000}. Now, consider a realization of $\tau=\hat{\tau}$ and suppose that we have a large number of agents $n$ observing a noisy variation of $\hat{\tau}$. Take for example the case of fire-fighting agents, and let $\hat{\tau}$ be the magnitude (including type, intensity, area, etc.) of the fire. Then, since the observations of the $n$ agents are independent given the value of $\tau$, they will be distributed according to $\mathcal{N}(\hat{\tau},\sigma^2)$ (given the value of $\tau$). Now consider the relative number of agents taking part in the activity given the realized magnitude parameter $\hat{\tau}$ and denote it by 
\begin{equation}\label{eqn:Nrel}
	N_{rel}(\hat{\tau}):=\frac{\#\text{agents with }x_i\geq \td}{n}.
\end{equation}

\nolinenumbers
\begin{figure}
\begin{framed}
\begin{theorem}\label{thrm:relativefrequency}
For the relative number of agents $N_{rel}(\hat{\tau})$, we have
\begin{equation}
\lim_{n\to\infty}N_{rel}(\hat{\tau})=\Phi(\frac{\hat{\tau}-\td}{\sigma^2})
\end{equation}
where $\Phi$ is the cumulative distribution function (cdf) of a standard Gaussian. 
\end{theorem}
\begin{proof}
Note that $N_{rel}(\hat{\tau})=\frac{\sum_{i=1}^n\mathbf{I}_{x_i\geq \td}}{n}$ where $\mathbf{I}_{\alpha\geq \beta}$ is the indicator function for $\alpha\geq \beta$. But note that for a given $\hat{\tau}$, $x_i$s are i.i.d.\ $\mathcal{N}(\hat{\tau},\sigma^2)$ random variables, and hence, $\mathbf{I}_{x_i\geq \td}$s are i.i.d.\ random variables with $E(\mathbf{I}_{x_i\geq \td})=\Phi(\frac{\hat{\tau}-\td}{\sigma^2})$. Therefore, by the Law of Large Numbers, it follows that:
\vspace{-5px}
\begin{align*}
\lim_{n\to\infty}N_{rel}(\hat{\tau})=\Phi(\frac{\hat{\tau}-\td}{\sigma^2}).
\end{align*}
\vspace{-35px}
\end{proof}
\vspace{-30px}
The final step to explain the prevalence of sigmoid functions in multiagent settings is to note that:
\begin{align*}
|\Phi(\frac{\hat{\tau}-\td}{\sigma^2})-\frac{1}{1+e^{-d(\frac{\hat{\tau}-\td}{\sigma^2})}}|\leq 0.01,
\end{align*}
for all $\hat{\tau}\in\R$ and some optimal value $d\approx 1.704$\cite{Camilli1994}. This means that the aggregate behaviour of the agents following deterministic threshold policies would (almost) follow the shape of a logistic sigmoid function whose drift is directly proportional to $\td$ and the slope  is inversely proportional to $\sigma^2$. 
\end{framed}
\end{figure}

\linenumbers
These results also hold when moving from a discrete to a continuous response threshold, which is shown in Box 2. Therefore, for large $n$, the relative frequency of agents taking part in the activity has a shape that follows the shape of the cdf of a standard Gaussian random variable, as seen in Fig.~2, i.e. agents may use deterministic threshold strategies but their aggregate behaviour would appear to an \textit{outside observer} as a continuous (sigmoid) threshold function.

Both the theorems and associated results presented in this paper suggest that global games can potentially describe a wide range of collective decision making scenarios, explaining the prevalence of sigmoid threshold policy in natural systems and artificial systems that preserve the beneficial BNE property. As the total number of agents increases, the response threshold strategy also allows the system to balance between exploitation and exploration. This, in turn, may lead to designing robust robot swarms that are flexible and can alter strategies for changing environmental parameters without requiring communication; a highly desired property that is often observed in natural swarms and is of great interest for engineered systems.

\nolinenumbers
\bibliography{references}

\begin{addendum}
 \item A. Kanakia and N. Correll have been supported by NSF CAREER grant \#1150223. We are grateful for this support. 
 \item[Competing Interests] The authors declare that they have no
competing financial interests.
 \item[Correspondence] Correspondence and requests for materials
should be addressed to Dr. Nikolaus Correll.~(email: ncorrell@colorado.edu).
\end{addendum}

%%
%% TABLES
%%
%% If there are any tables, put them here.
%%

\newpage
\section*{Supplementary Material}

\setcounter{lemma}{0}
\begin{lemma}
Let $s=(s_1,\ldots,s_n)$ be a strategy profile consisting of threshold strategies for a task with  concurrent benefit. Let $\tilde{s}_i=BR(s_{-i})$. Then $\tilde{s}_i$ is a threshold policy. 
\end{lemma} 

\begin{proof}
We first show that if for some observation $x_i=x$, we have $BR(s_{-i})(x)=\tilde{s}_i(x)=1$, then $\tilde{s}_i(y)=1$ for $y\geq x$. To show this,  we note that $P(x_j\geq \tau_j\mid x_i=x)$ is an increasing function of $x$ as $x_j-x_i$ is a normally distributed random variable. Therefore, using the monotone property of concurrent tasks and the fact that $x_i=\tau+\eta_i$, we conclude that:
\vspace{-5px}
\begin{align*}
&E(u_i(1,\sum_{j\not=i}s_j(x_j),\tau)\mid x_i=y)\\ 
&\qquad-E(u_i(0,\sum_{j\not=i}s_j(x_j),\tau)\mid x_i=y)\\ 
&>E(u_i(1,\sum_{j\not=i}s_j(x_j),\tau)\mid x_i=x)\\
&\qquad-E(u_i(0,\sum_{j\not=i}s_j(x_j),\tau)\mid x_i=x)\geq 0.
\end{align*}
Therefore $\tilde{s}_i(y)=1$. Similarly, if for some value of $x$, we have $\tilde{s}_i(x)=0$, then it follows that $\tilde{s}_i(y)=0$ for $y\leq x$. Therefore, $\tilde{s}_i$ would be a threshold policy.  
\end{proof}


\begin{lemma}
The mapping $L$ that maps the threshold values of threshold strategies to the threshold values of the best-response strategies is a continuous mapping. 
\end{lemma}

\begin{proof}
Let $x_{-i}=(x_1,\ldots,x_{i-1},x_{i+1},\ldots,x_n)$ be the vector of observations of $n-1$ agents except the $i^{\text{th}}$ agent. Note that the vector $(x_{-i},\tau)$ given $x_i=x$ is a normally distributed random vector with some continuous density function $f_{x}(x_{-i},\tau)$. Now, let $\{\alpha(k)\}$ be a sequence in $\R^n$ that is converging to $\alpha\in\R^n$. Let $\{\beta(k)\}$ be the sequence of thresholds corresponding to the best-response strategy of the strategy with threshold vector $\alpha(k)$. Let $s$ be the threshold strategy corresponding to the threshold vector $\alpha$ and let $\alpha^*$ be the threshold policy corresponding to the $BR(\alpha)$. By the definition of the best-response strategy, $\beta_i(k)$ is a point where 
\begin{align*}
&\int_{\R^{n}}f_{\beta(k)}(z,t)\left(u_i(1,\sum_{j\not=i}u^{\alpha_j(k)}(x_j),\tau)\right.\\
&\qquad\left.-u_i(0,\sum_{j\not=i}u^{\alpha_j(k)}(x_j),\tau)\right)d(z\times t)=0.
\end{align*}
Using the fact that $f$ has a Gaussian distribution and is continuous on all its arguments and the fact that $|u_i(\cdot,\cdot,\tau)|\leq \tau^p$, by taking the limit $k\to\infty$ and the dominated convergence theorem:
\begin{align*}
&\int_{\R^{n}}f_{\beta}(z,t)(u_i(1,\sum_{j\not=i}u^{\alpha_j}(x_j),\tau)\\ 
&\qquad-u_i(0,\sum_{j\not=i}u^{\alpha(k)}(x_j),\tau))d(z\times t)=0,
\end{align*}
where $u^{r}$ is a threshold strategy with threshold $r$. Therefore, the $\lim_{k\to\infty}L(\alpha(k))=L(\alpha)$ for a sequence $\{\alpha(k)\}$ that is converging to $\alpha$.
\end{proof}


\end{document}
