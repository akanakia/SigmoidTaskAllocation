%% Template for a preprint Letter or Article for submission
%% to the journal Nature.
%% Written by Peter Czoschke, 26 February 2004
%%
\documentclass{nature}

%% make sure you have the nature.cls and naturemag.bst files where
%% LaTeX can find them
\bibliographystyle{naturemag}

\title{Optimal task allocation in systems with limited communication}

%% Notice placement of commas and superscripts and use of &
%% in the author list

\author{Anshul Kanakia$^{1,2}$, Nikolaus Correll$^{1,2}$ \& Behrouz Touri$^{1,3}$}


\begin{document}

\maketitle

\begin{affiliations}
 \item College of Engineering and Applied Sciences, University of Colorado, Boulder, USA. 
 \item Dept. of Computer Science.
 \item Dept. of Electrical, Computer \& Energy Engineering.
\end{affiliations}

\begin{abstract}
Task allocation in systems that are subject to communication constraints is ubiquitous in nature at all scales, ranging from cellular systems, social insects to people. Although the ability to communicate among agents to, for example, vote \cite{Conradt2003} increases with their complexity, inter-agent communication might not always possible. Formalizing such systems as a global game, a concept from the field of game theory, reveals that a simple threshold policy leads to a Bayesian Nash Equilibrium, that is an optimal assignment in the absence of communication between agents. While this result provides an hypothesis about the inner workings of a wide range of systems in which limited communication between agents is likely, it provides a formal explanation for threshold-based task allocation in social insects. In particular, we show how noise in perception in such systems leads to the observation of sigmoidal response threshold functions \cite{Bonabeau2000}, and how the resulting trade-off between exploration and exploitation can be used to design engineered systems ranging from swarm robotics \cite{Gerkey2004} to smart composites \cite{McEvoy2015}. 
\end{abstract}

We are considering the following scenario. There are a group of individual agents attempting to perform some collaborative task. All agents observe a stimulus of a certain magnitude associated with this task. Observation of this stimulus is subject to noise, and therefore subjective to each agent. Agents do not share any information. All agents decide, for themselves, whether or not to engage in the collaborative task. When a critical mass of agents willing to take part in the task is achieved, the task is successfully attempted, otherwise the attempt automatically fails. 

We have seen situations like this in a number of different fields including ethology, neurology \cite{Suzuki2015}, sociology, economics \cite{Morris2000}, and computer science. 
%%%% CAN YOU FILL IN THE ABOVE LIST WITH REFERENCES?
While collaboration in social organisms has been a topic of study for centuries, the underlying mechanisms of this process still elude researchers. A broadening in thinking of the way division of labor is achieved in social insect colonies is discussed in \cite{Gordon1996}. Here, the idea that individuals in a swarm decide what task to perform using both internal and external triggers in a dynamically shifting environment is first cultivated. In contrast, prior models outlined in \cite{Gordon1996} first attributed task allocation mechanisms to innate physiological differences in individuals (such as age and size, in research from the 1960s/70s) while later models tended to focus heavily on external agent stimulus (such as pheromone levels, seen in research from the 1980s). It is clear now that a combination of both internal and external parameters in essential for understanding the underlying mechanisms governing task-allocation in multi-agent systems. 

This problem setup can be formulated as a \emph{global game}\cite{Carlsson1993}. Global games involves the study of games where individual agents' utilities for taking an action depend not only on a shared underlying noisy signal, but also their imperfect beliefs of other agents' strategies. A classical example is a bank-run during a financial crisis\cite{Morris2000}. Here, agents engage in a bank-run based on their own, noisy, perception of the global financial situation as well as on their assessment how likely other players are to withdraw their savings. Although the classical global game assumes each agent to predict the other agents' behaviors, it turns out that agents can reach an equilibrium also without having this capacity\cite{Morris2000}, making this theory applicable to a much wider range of systems.  

%In the reminder of this paper, we refer to the commonly observed signal as a \emph{stimulus} and its numerical value its \emph{magnitude}. The magnitude term is an abstraction of the underlying quality of any task, i.e. the demand for resources to complete said task. 

%%% WE SHOULD ADD THE MATH FROM THE IROS PAPER HERE.

%Indeed, response-threshold models have become popular due to their simplicity and accuracy with which they successfully describe collaborative task behavior in multi-agent systems. While extensive phenomenological evidence exists supporting the agreement between response-threshold models and successful task-assignment, there has so far been no logical argument for their extensive use.

We have therefore shown that a simple response threshold is sufficient to achieve a Bayes Nash Equilibrium, that is there is no other strategy that will, on average, improve an individual agents reward. This insight is consistent with phenomenological observations in social insects, which are believed to use response thresholds for coordination. For example, Robinson\cite{Robinson1987} provides first empirical evidence for hormone regulated response-threshold labor division in honey bees. A number of other ethology research groups subsequently show similar hormone regulation based task-allocation processes in ants and termites providing further evidence of the ubiquitous nature of the response-threshold process. Roboticists have started using these models to engineer swarm systems\cite{Krieger2000}. 
Bonabeau \cite{Bonabeau1996} introduces the \emph{fixed} response-threshold model for division of labor in social insects which is later extended to incorporate the notion of \emph{dynamic} response-thresholds for task-allocation which overcomes a number of issues with the fixed model \cite{Theraulaz1998}.

Yet, observations in ethology suggest a Sigmoidal threshold function\cite{Bonabeau1996}. We argue that this behavior is a direct result of using a simple discrete threshold under the influence of perception noise. Indeed, one can show that 

%%% ADD THE SECOND THEOREM HERE

A sigmoidal threshold function therefore naturally emerges if there is noise in observing the stimulus. When engineering a system, this noise can be controlled and enhanced using computation, allowing artificial agents to trade between exploitation and exploration. 



%%% WHAT IS THE KEY PROBLEM WITH COLLABORATION? AGREEING ON WHEN TO DO IT REQUIRES A COMMON STIMULULUS AND COORDINATION
%Collaboration is a fundamental necessity in any society. Tasks such as foraging, farming, hunting, and construction that were essential building blocks of human civilization greatly benefited from collaboration between multiple individuals. 

%This sheds light on the key complexity for collaboration; a common perceived stimulus must drive a large group of similar agents to work together to achieve a common goal without any central control directive. This is an extremely lofty goal to achieve for artificial systems such as robot swarms but what is most astounding this the banality with which it is observed in social natural systems such as bee hives and ant colonies. Social insects have evolved to facilitate collaboration in a decentralized manner. The problem of allocating tasks to a group of agents is solved in a completely distributed fashion within insect swarms. While a central queen organism often exists in insect colonies, this organism's duty is purely reproductive in nature, not administrative.


%%% THE PROBLEM HAS NOT ONLY BEEN STUDIED BY ETHOLOGISTS FOR ANTS BUT ALSO BY NEUROLOGISTS AND ECONOMISTS (GAME THEORY). WHO ELSE STUDIES TASK ALLOCATION IF IT IS REALLY SO IMPORTANT?

%%% WHAT ARE THE COMMONALITIES AMONG ALL THE CONSENSUS PROBLEMS / WHAT CLASS OF PROBLEMS ARE WE INTERESTED IN. THIS SHOULD MATCH THE GAME THEORY DEFINITION. THERE ARE MANY. WHERE HAS THE RESPONSE THRESHOLD ALREADY BEEN OBSERVED AND WHERE IS IT USED? ETHOLOGY AND ROBOTICS. THEY ARE DIFFERENT THAN WHAT GAME THEORY ASSUMES, HERE IS WHY: NOISE. 
%Gerkey \cite{Gerkey2004} Provides a formal taxonomy of the state of the art for task-allocation in multi-robot systems. The axes of single-task robots performing multi-robot tasks with instantaneous and time-extended assignments is of particular interest as it envelops many of the application domains discussed in this paper.


% MAIN CLAIM: THERE ARE COUNTLESS SYSTMES THAT MUST USE RESPONSE THRESHOLDS, BECAUSE THEY ARE THE BEST WAY TO DO IT IN THIS CASE.

 
\cite{Bonabeau2000}

\cite{Krieger2000}

\cite{Kube2000}

\cite{Pynadath2002}

\cite{Conradt2003}

\cite{Mataric2003}

\cite{Gerkey2003}

\cite{Conradt2005}

%Raafat et al. \cite{Raafat2009} discuss the underlying control mechanism for distributed coordination in a multi-agent system. They split the study of this mechanism into two levels; transmission-based and pattern-based control. 

\cite{Yoshida2010}






%% Put the bibliography here, most people will use BiBTeX in
%% which case the environment below should be replaced with
%% the \bibliography{} command.
\bibliography{references}

\begin{addendum}
 \item Put acknowledgements here.
 \item[Competing Interests] The authors declare that they have no
competing financial interests.
 \item[Correspondence] Correspondence and requests for materials
should be addressed to Dr. Nikolaus Correll.~(email: nikolaus.correll@colorado.edu).
\end{addendum}

%%
%% TABLES
%%
%% If there are any tables, put them here.
%%

\end{document}
