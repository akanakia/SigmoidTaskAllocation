%% Template for a preprint Letter or Article for submission
%% to the journal Nature.
%% Written by Peter Czoschke, 26 February 2004
%%

\documentclass{nature}

%% make sure you have the nature.cls and naturemag.bst files where
%% LaTeX can find them
\bibliographystyle{naturemag}

\title{Mathematical Origins of the Sigmoid Response-Threshold Function}

%% Notice placement of commas and superscripts and use of &
%% in the author list

\author{Anshul Kanakia$^{1,2}$, Nikolaus Correll$^{1,2}$ \& Behrouz Touri$^{1,3}$}


\begin{document}

\maketitle

\begin{affiliations}
 \item College of Engineering and Applied Sciences, University of Colorado, Boulder, USA. 
 \item Dept. of Computer Science.
 \item Dept. of Electrical, Computer \& Energy Engineering.
\end{affiliations}

\begin{abstract}
Response-threshold models have been used to model task-allocation in insect colonies since the 1980s. More recently, these models have been picked-up by swarm roboticists to solve the problem of task-assignment in multi-agent systems (MAS). So far, the justification for the  widespread use of response-threshold models in both communities has been phenomenological in nature, i.e. the relative accuracy with which they predict observed insect colony behavior. We present a new game theory based approach to provide a rational argument for their success   by showing that independent threshold based task-allocation by agents results in a system-wide Bayes Nash Equilibrium.
\end{abstract}

%% Paper content goes here. Sections are not allowed in Letters to Nature
Collaboration is a fundamental necessity in any society. Tasks such as foraging, farming, hunting, and construction that were essential building blocks of human civilization greatly benefited from collaboration between multiple individuals. Social insects have evolved to facilitate collaboration in a decentralized manner. The problem of allocating tasks to a group of agents is solved in a completely distributed fashion within insect swarms such as ant colonies and bee hives. While a central queen organism often exists in insect colonies, this organism's duty is purely reproductive in nature, not administrative. 

Ethologists have studied the problem of distribution of labor in social insects extensively and have found that a class of models called response-threshold models explain observed behavior very well. Swarm roboticists have adopted this model of distributed task-assignment and adapted it for use with artificially engineered systems. While extensive phenomenological evidence exists supporting the agreement between response-threshold models and successful task-assignment, there has so far been no logical argument for their extensive use. 

 
%% Put the bibliography here, most people will use BiBTeX in
%% which case the environment below should be replaced with
%% the \bibliography{} command.
\bibliography{references}

\begin{addendum}
 \item Put acknowledgements here.
 \item[Competing Interests] The authors declare that they have no
competing financial interests.
 \item[Correspondence] Correspondence and requests for materials
should be addressed to A.B.C.~(email: myaddress@nowhere.edu).
\end{addendum}

%%
%% TABLES
%%
%% If there are any tables, put them here.
%%

\end{document}
