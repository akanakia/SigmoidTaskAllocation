%%%%%%%%%%%%%%%%%%%%%%%%%%%%%%%%%%%%%%%%%%%%%%%%%%%%%%%%%%%%%%%%%%%%%%%%%%%%%%%%
%2345678901234567890123456789012345678901234567890123456789012345678901234567890
%        1         2         3         4         5         6         7         8

\documentclass{sage}

% Usual setup packages
\usepackage{listings} % For including source code with highlighting
\usepackage{hyperref} % For better hyper-link integration
\usepackage[bottom]{footmisc} % places footnotes at page bottom

% Packages for verbatim text blocks
\usepackage{alltt} % Package for including math in verbatim text
\usepackage{fancyvrb}

% Packages for math symbols and other mathey things
\usepackage{amsthm}
\usepackage{amsmath}
\usepackage{amsfonts}
\usepackage{amssymb}

% Packages for including pseudo-code
\usepackage{algorithmicx}
\usepackage{algorithm}
\usepackage{algpseudocode}

% Packages that handle tables, figures and other floats
\usepackage{tabularx}
\usepackage{multirow}
\usepackage{float} % To make floats movable
\usepackage{subcaption}
\usepackage[table]{xcolor}

% Packages for drawing graphs, FSMs, etc.
\usepackage{pgf}
\usepackage{tikz}
\usetikzlibrary{shapes,arrows,calc,fit,positioning,shapes.symbols,shapes.callouts,patterns,automata}

% Remove red boxes around refs
\hypersetup{
    colorlinks,
    citecolor=black,
    filecolor=black,
    linkcolor=black,
    urlcolor=blue
}

% ------------------------------ CUSTOM MACROS ------------------------------------
% Nice little macro for adding a comment box. Include incrementing comment numbers.
\newcounter{comcount}
\setcounter{comcount}{0}
\newcommand{\mycomment}[1]
{
\refstepcounter{comcount}
\smallskip\noindent\fbox{\parbox{\linewidth}{\emph{Comment \arabic{comcount}} : \small{#1}}} 
}

\newcommand{\sig}{\mathcal{S}}
\newcommand{\ceil}[1]{\lceil#1\rceil}
\newcommand{\xm}{x_{\hat{m}}}

%%%Article information%%%%%%%%%%%%%%%%%%%%%%%%%%%%%%%%
\journal{International Journal of Robotics Research}
\volume{000}
\issue{00}
\copyrightline{$\copyright$ Copyright}
\firstpage{1}
\lastpage{13}
\doi{doi number}
\articletype{Article type}
\pubyear{2015}
%%%%%%%%%%%%%%Article information%%%%%%%%%%%%%%%%%%%%%%%%

\begin{document}
\title{Maximizing Swarm System Utility for Time-Varying Tasks}
\author{Anshul Kanakia and Nikolaus Correll}
\address{Department of Computer Science,
	University of Colorado, Boulder, USA}

\maketitle

%%%%%%%%%%%%%%%%%%%%%%%%%%%%%%%%%%%%%%%%%%%%%%%%%%%%%%%%%%%%%%%%%%%%%%%%%%%%%%%%
\begin{abstract}
Abstract goes here\ldots
\end{abstract}
\keywords{Swarm Robotics, Multi-Agent Systems, Collaboration, Task Allocation, Utility Theory, Optimization}

%%%%%%%%%%%%%%%%%%%%%%%%%%%%%%%%%%%%%%%%%%%%%%%%%%%%%%%%%%%%%%%%%%%%%%%%%%%%%%%%
\section{Introduction}
There has been considerable research done using swarm robot systems to tackle tasks with unknown utility such as firefighting. Time often plays an important role in such situations, e.g. fires tend to spread at an exponential rate if not contained quickly. Thus, the overall utility of the system also varies with time. We describe a process that maximizes time-varying utility of a swarm robot system for accomplishing time-varying tasks, through the use of sigmoid threshold functions for collaboration decisions.

\section{Defining System Utility}



%%%%%%%%%%%%%%%%%%%%%%%%%%%%%%%%%%%%%%%%%%%%%%%%%%%%%%%%%%%%%%%%%%%%%%%%%%%%%%%%
%%%%%%%%%   The Bibliography, if any   %%%%%%%%%
\bibliographystyle{harvard}		% or "siam", or "alpha", etc.
%\nocite{*}						% list all refs in database, cited or not
\bibliography{../refworks}
\end{document}